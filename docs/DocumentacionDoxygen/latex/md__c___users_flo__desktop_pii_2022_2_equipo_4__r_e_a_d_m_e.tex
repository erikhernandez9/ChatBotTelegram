REFLEXIONES\+:

Durante el transcurso del proyecto nos enfrentamos a muchos desafíos que nos enseñaron a no rendirnos y ser constantes en nuestro trabajo. Aunque al principio nos costó iniciar, fuimos \char`\"{}buscando la vuelta\char`\"{} y conseguimos adaptarnos a C\#.

Luego de cada entrega nos sentamos a pensar en qué nos equivocamos, cómo pudimos hacerlo mejor y qué logramos hacer correctamente. Creemos que es muy importante darle el lugar que merecen nuestros logros, ya que siempre dimos lo mejor de nosotros y pusimos toda nuestra energía en completar las tareas a tiempo de la mejor forma posible.

Nos parece que los contenidos dados en el curso y los materiales brindados por los profesores fueron los pilares básicos para la realización del proyecto, ya que son herramientas que han sido cruciales para el desarrollo del curso y además son recursos que nos serán útiles a lo largo de toda nuestra carrera. Además creemos que la variedad de actividades propuestas estuvieron acorde a lo que necesitamos. Las aptitudes que desarrollamos en esas actividades nos hicieron crecer y darnos cuenta de cosas que nos fueron muy útiles para el trabajo en el bot, como el Pipes and Filters.

Creemos que entre los desafíos del proyecto se encuentran comprender cómo optimizar el código, aplicar los patrones y sobre todo implementar el bot. El desafío mayor fue el tiempo, ya que aunque trabajamos mucho siempre, siempre fuimos contrarreloj.

Sin duda creemos que realizar este proyecto fue una experiencia muy enriquecedora que nos deja aprendizajes muy valiosos sobre el trabajo en equipo, la constancia, el llegar a entregas y cosas teóricas en general, que será muy útil.

//\+\_\+\+\_\+\+\_\+\+\_\+\+\_\+\+\_\+\+\_\+\+\_\+\+\_\+\+\_\+\+\_\+\+\_\+\+\_\+\+\_\+\+\_\+\+\_\+\+\_\+\+\_\+\+\_\+\+\_\+\+\_\+\+\_\+\+\_\+\+\_\+\+\_\+\+\_\+\+\_\+\+\_\+\+\_\+\+\_\+\+\_\+\+\_\+\+\_\+\+\_\+\+\_\+\+\_\+\+\_\+\+\_\+\+\_\+\+\_\+\+\_\+\+\_\+\+\_\+\+\_\+\+\_\+\+\_\+\+\_\+\+\_\+\+\_\+\+\_\+\+\_\+\+\_\+\+\_\+\+\_\+\+\_\+\+\_\+\+\_\+\+\_\+\+\_\+\+\_\+\+\_\+\+\_\+\+\_\+\+\_\+\+\_\+\+\_\+\+\_\+\+\_\+\+\_\+\+\_\+\+\_\+\+\_\+\+\_\+\+\_\+\+\_\+\+\_\+\+\_\+\+\_\+\+\_\+\+\_\+\+\_\+\+\_\+\+\_\+\+\_\+\+\_\+\+\_\+\+\_\+\+\_\+\+\_\+\+\_\+\+\_\+\+\_\+\+\_\+\+\_\+\+\_\+\+\_\+//

ASIGNACIÓN de CLASES\+:
\begin{DoxyItemize}
\item Verificar\+String \+: Erik
\item \mbox{\hyperlink{class_usuario}{Usuario}} \+: Renzo
\item Singleton \+: Florencia
\item \mbox{\hyperlink{class_persona}{Persona}} \+: Emiliano
\item Oferta\+Laboral \+: Emiliano
\item \mbox{\hyperlink{class_notificaciones}{Notificaciones}} \+: Erik
\item \mbox{\hyperlink{class_lista_trabajadores}{Lista\+Trabajadores}} \+: Florencia
\item \mbox{\hyperlink{class_lista_empleadores}{Lista\+Empleadores}} \+: Emiliano
\item \mbox{\hyperlink{class_lista_administradores}{Lista\+Administradores}} \+: Renzo
\item Library\+Exception \+: Erik
\item IUser \+: Florencia
\item \mbox{\hyperlink{interface_i_json_convertible}{IJson\+Convertible}} \+: Emiliano
\item Exception\+Constructor \+: Florencia
\item Contratacion \+: Renzo
\item Categoria \+: Emiliano
\item Calificacion \+: Erik
\item Calcular\+Reputacion \+: Florencia
\item Trabajador \+: Emiliano
\item Empleador \+: Erik
\item Administrador \+: Florencia
\item Location\+Api\+Client \+: Renzo
\item Location \+: Erik
\item IDistance\+Calculator \+: Emiliano
\item IAdress\+Finder \+: Emiliano
\item Distance\+Calculator \+: Florencia
\item Distance \+: Erik
\item Catalogo\+String\+Builder \+: Florencia
\item Adress\+Finder \+: Renzo
\item Address\+Result \+: Emiliano
\item Start\+Handler \+: Florencia
\item Photo\+Handler \+: Erik
\item Mostrar\+Catalogo\+Ofertas\+Handler \+: Florencia
\item IHandler \+: Erik
\item Help\+Handler \+: Emiliano
\item Hello\+Handler \+: Renzo
\item Good\+Bye\+Handler \+: Renzo
\item Eliminar\+Trabajador\+Handler \+: Erik
\item Eliminar\+Oferta\+Handler \+: Emiliano
\item Eliminar\+Empleador\+Handler \+: Florencia
\item Eliminar\+Categoria\+Handler \+: Erik
\item Distance\+Handler \+: Renzo
\item Crear\+Oferta\+Laboral\+Handler \+: Emiliano
\item Crear\+Contratacion\+Handler \+: Erik
\item Calificar\+Trabajador\+Handler \+: Florencia
\item Calificar\+Empleador\+Handler \+: Florencia
\item Calificacion\+Handler \+: Renzo
\item Base\+Handler \+: Emiliano
\item Agregar\+Trabajador\+Handler \+: Emiliano
\item Agregar\+Empleador\+Handler \+: Erik
\item Adress\+Handler \+: Erik
\item Catalogo\+Ofertas \+: Emiliano
\item Catalogo\+Contratacion \+: Florencia
\item Catalogo\+Categorias \+: Renzo
\item Adress\+Handler\+Tests \+: Erik
\item Catalogo\+Categorias\+Tests \+: Emiliano
\item Catalogo\+Contrataciones\+Tests \+: Florencia
\item Catalogo\+Ofertas\+Tests \+: Emiliano
\item Contratacion\+Tests \+: Florencia
\item Empleador\+Test \+: Florencia
\item Handler\+Distancia\+Tests \+: Erik
\item Hello\+Handler\+Tests \+: Renzo
\item Lista\+Administradores\+Test \+: Erik
\item Lista\+Empleadores\+Test \+: Emiliano
\item Lista\+Trabajadores\+Test \+: Renzo
\item Responsabilidades\+Tests \+: Erik
\item Trabajador\+Tests \+: Florencia
\end{DoxyItemize}

//\+\_\+\+\_\+\+\_\+\+\_\+\+\_\+\+\_\+\+\_\+\+\_\+\+\_\+\+\_\+\+\_\+\+\_\+\+\_\+\+\_\+\+\_\+\+\_\+\+\_\+\+\_\+\+\_\+\+\_\+\+\_\+\+\_\+\+\_\+\+\_\+\+\_\+\+\_\+\+\_\+\+\_\+\+\_\+\+\_\+\+\_\+\+\_\+\+\_\+\+\_\+\+\_\+\+\_\+\+\_\+\+\_\+\+\_\+\+\_\+\+\_\+\+\_\+\+\_\+\+\_\+\+\_\+\+\_\+\+\_\+\+\_\+\+\_\+\+\_\+\+\_\+\+\_\+\+\_\+\+\_\+\+\_\+\+\_\+\+\_\+\+\_\+\+\_\+\+\_\+\+\_\+\+\_\+\+\_\+\+\_\+\+\_\+\+\_\+\+\_\+\+\_\+\+\_\+\+\_\+\+\_\+\+\_\+\+\_\+\+\_\+\+\_\+\+\_\+\+\_\+\+\_\+\+\_\+\+\_\+\+\_\+\+\_\+\+\_\+\+\_\+\+\_\+\+\_\+\+\_\+\+\_\+\+\_\+\+\_\+\+\_\+\+\_\+\+\_\+\+\_\+\+\_\+\+\_\+//


\begin{DoxyItemize}
\item USER STORIES\+:
\end{DoxyItemize}

Cómo administrador, quiero poder indicar categorías sobre las cuales se realizarán las ofertas de servicios para que de esa forma, los trabajadoras puedan clasificarlos. \mbox{[}CUMPLE\mbox{]}

Como administrador, quiero poder dar de baja ofertas de servicios, \{avisando al oferente para que de esa forma, pueda evitar ofertas inadecudas\}. \mbox{[}CUMPLE\mbox{]}

Como trabajador, quiero registrarme en la plataforma, indicando mis datos personales e información de contacto para que de esa forma, pueda proveer información de contacto a quienes quieran contratar mis servicios. \mbox{[}CUMPLE\mbox{]}

Como trabajador, quiero poder hacer ofertas de servicios; mi oferta indicará en qué categoría quiero publicar, tendrá una descripción del servicio ofertado, y un precio para que de esa forma, mis ofertas sean ofrecidas a quienes quieren contratar servicios. \mbox{[}CUMPLE\mbox{]}

Como empleador, quiero registrarme en la plataforma, indicando mis datos personales e información de contacto para que de esa forma, pueda proveer información de contacto a los trabajadores que quiero contratar. \mbox{[}CUMPLE\mbox{]}

Como empleador, quiero buscar ofertas de trabajo, opcionalmente filtrando por categoría para que de esa forma, pueda contratar un servicio. \mbox{[}CUMPLE\mbox{]}

Como empleador, quiero ver el resultado de las búsquedas de ofertas de trabajo ordenado en forma {\itshape ascendente} de {\itshape distancia a mi ubicación}, es decir, las más cercanas primero para que de esa forma, pueda poder contratar un servicio. \mbox{[}CUMPLE\mbox{]}

Como empleador, quiero ver el resultado de las búsquedas de ofertas de trabajo ordenado en forma {\itshape descendente} por reputación, es decir, las {\itshape de mejor reputación primero} para que de esa forma, pueda contratar un servicio. \mbox{[}CUMPLE\mbox{]}

Como empleador, quiero poder contactar a un trabajador para que de esa forma pueda, contratar una oferta de servicios determinada. \mbox{[}CUMPLE\mbox{]}

Como {\itshape trabajador}, quiero poder calificar a un empleador; el empleador me tiene que calificar a mi también, \{si no me califica en un mes, la calificación será neutral, para que de esa forma pueda definir la reputación de mi empleador.\} \mbox{[}CUMPLE\mbox{]}

Como {\itshape empleador}, quiero poder calificar a un trabajador; el trabajador me tiene que calificar a mi también, \{si no me califica en un mes, la calificación será neutral, para que de esa forma, pueda definir la reputación del trabajador.\} \mbox{[}CUMPLE\mbox{]}


\begin{DoxyItemize}
\item Como trabajador, quiero poder saber la reputación de un empleador que me contacte para que de esa forma, poder decidir sobre su solicitud de contratación. \mbox{[}CUMPLE\mbox{]}
\end{DoxyItemize}

//\+\_\+\+\_\+\+\_\+\+\_\+\+\_\+\+\_\+\+\_\+\+\_\+\+\_\+\+\_\+\+\_\+\+\_\+\+\_\+\+\_\+\+\_\+\+\_\+\+\_\+\+\_\+\+\_\+\+\_\+\+\_\+\+\_\+\+\_\+\+\_\+\+\_\+\+\_\+\+\_\+\+\_\+\+\_\+\+\_\+\+\_\+\+\_\+\+\_\+\+\_\+\+\_\+\+\_\+\+\_\+\+\_\+\+\_\+\+\_\+\+\_\+\+\_\+\+\_\+\+\_\+\+\_\+\+\_\+\+\_\+\+\_\+\+\_\+\+\_\+\+\_\+\+\_\+\+\_\+\+\_\+\+\_\+\+\_\+\+\_\+\+\_\+\+\_\+\+\_\+\+\_\+\+\_\+\+\_\+\+\_\+\+\_\+\+\_\+\+\_\+\+\_\+\+\_\+\+\_\+\+\_\+\+\_\+\+\_\+\+\_\+\+\_\+\+\_\+\+\_\+\+\_\+\+\_\+\+\_\+\+\_\+\+\_\+\+\_\+\+\_\+\+\_\+\+\_\+\+\_\+\+\_\+\+\_\+\+\_\+\+\_\+\+\_\+\+\_\+\+\_\+\+\_\+\+\_\+//

MATERIAL\+:

recopilamos información de patrones y design patterns de\+:
\begin{DoxyItemize}
\item \href{https://refactoring.guru/es}{\texttt{ https\+://refactoring.\+guru/es}}
\item \href{https://en.wikipedia.org/wiki/Plain_old_CLR_object-}{\texttt{ https\+://en.\+wikipedia.\+org/wiki/\+Plain\+\_\+old\+\_\+\+CLR\+\_\+object-\/}}
\item \href{https://jbravomontero.files.wordpress.com/2012/12/solid-y-grasp-buenas-practicas-hacia-el-exito-en-el-desarrollo-de-software.pdf}{\texttt{ https\+://jbravomontero.\+files.\+wordpress.\+com/2012/12/solid-\/y-\/grasp-\/buenas-\/practicas-\/hacia-\/el-\/exito-\/en-\/el-\/desarrollo-\/de-\/software.\+pdf}} ... y materiales del curso.
\end{DoxyItemize}

//\+\_\+\+\_\+\+\_\+\+\_\+\+\_\+\+\_\+\+\_\+\+\_\+\+\_\+\+\_\+\+\_\+\+\_\+\+\_\+\+\_\+\+\_\+\+\_\+\+\_\+\+\_\+\+\_\+\+\_\+\+\_\+\+\_\+\+\_\+\+\_\+\+\_\+\+\_\+\+\_\+\+\_\+\+\_\+\+\_\+\+\_\+\+\_\+\+\_\+\+\_\+\+\_\+\+\_\+\+\_\+\+\_\+\+\_\+\+\_\+\+\_\+\+\_\+\+\_\+\+\_\+\+\_\+\+\_\+\+\_\+\+\_\+\+\_\+\+\_\+\+\_\+\+\_\+\+\_\+\+\_\+\+\_\+\+\_\+\+\_\+\+\_\+\+\_\+\+\_\+\+\_\+\+\_\+\+\_\+\+\_\+\+\_\+\+\_\+\+\_\+\+\_\+\+\_\+\+\_\+\+\_\+\+\_\+\+\_\+\+\_\+\+\_\+\+\_\+\+\_\+\+\_\+\+\_\+\+\_\+\+\_\+\+\_\+\+\_\+\+\_\+\+\_\+\+\_\+\+\_\+\+\_\+\+\_\+\+\_\+\+\_\+\+\_\+\+\_\+\+\_\+\+\_\+\+\_\+//

HANDLERS\+:


\begin{DoxyItemize}
\item /start (inicio de sesion) -\/-\/---\texorpdfstring{$>$}{>} es una user story
\item Location\+API para ubicar al usuario
\item Calificación de trabajador (en caso de pendiente)
\item Calificación de empleador (en caso de pendiente)
\item Crear Oferta
\item Lista Ofertas por precio
\item Lista Ofertas por distancia
\item Lista Ofertas por calificación
\item Mostrar lista de Categorías
\item Mostrar contrataciones activas
\item Mostrar historial de contrataciones
\item Consulta de Ofertas por categoría
\item Contratación
\item Agregar Oferta (trabajadores)
\item Agregar Categoría (administradores)
\item Eliminar Trabajador
\item Eliminar Empleador
\item Eliminar Categoría
\item Ayuda
\item Agregar Trabajador
\item Agregar Empleador
\end{DoxyItemize}

//\+\_\+\+\_\+\+\_\+\+\_\+\+\_\+\+\_\+\+\_\+\+\_\+\+\_\+\+\_\+\+\_\+\+\_\+\+\_\+\+\_\+\+\_\+\+\_\+\+\_\+\+\_\+\+\_\+\+\_\+\+\_\+\+\_\+\+\_\+\+\_\+\+\_\+\+\_\+\+\_\+\+\_\+\+\_\+\+\_\+\+\_\+\+\_\+\+\_\+\+\_\+\+\_\+\+\_\+\+\_\+\+\_\+\+\_\+\+\_\+\+\_\+\+\_\+\+\_\+\+\_\+\+\_\+\+\_\+\+\_\+\+\_\+\+\_\+\+\_\+//

{\itshape GRASP Y SOLID\+: PATRONES Y PRINCIPIOS.}


\begin{DoxyItemize}
\item GRASP\+: \begin{DoxyVerb}  - Expert: Experto en información nos dice que la responsabilidad de la creación de un objeto o la implementación de un método, debe recaer sobre la clase que conoce toda la información necesaria para crearlo o ejecutarlo.
  - Creator: nos ayuda a identificar quién debe ser el responsable de la creación o instanciación de nuevos objetos o clases. La nueva instancia podrá ser creada si: Contiene o agrega la clase, tiene la información necesaria para realizar la creación del objeto, o usa directamente las instancias creadas del objeto.
  - Polimorfismo:  polimorfismo es permitir que varias clases se comporten de manera distinta dependiendo del tipo que sean. Siempre que se haga una responsabilidad que dependa de un tipo, utilizaremos polimorfismo.
  - Low coupling: Tener las clases lo menos "conectadas" entre sí que se pueda, para tener la mínima repercusión posible en el resto de clases por si se modifica alguna. Potenciando la reutilización y disminuyendo la dependencia.
  - High Cohesion: Nos dice que la información que almacena una clase debe de ser coherente y debe estar, en la medida de lo posible,          relacionada con la clase.  
\end{DoxyVerb}

\item SOLID\+: \begin{DoxyVerb}  - SRP:  Una clase debería concentrarse sólo en hacer una cosa para que cuando cambie algo dicho cambio sólo afecte a esa clase por una razón. (single responsability)
  - DIP: para conseguir robustez y flexibilidad y para posibilitar la reutilización el código depende de abstracciones y no de concreciones, utilizar muchas interfaces y muchas clases abstractas.
  - ISP: Interface segregation principle. Mantener las interfaces pequeñas y cohesivas para que puedan coexistir unas con otras. (no depender de interfaces con cosas que no utilicen) 
  - OCP: Cambia el comportamiento de una clase mediante herencia, polimorfismo y composición. Los objetos o entidades deben estar abiertos para la extensión pero cerrados para la modificación.
  - LSP: Las subclasses deben comportarse adecuadamente cuando sean usadas en lugar de sus clases base. 
\end{DoxyVerb}

\item OTROS PATRONES\+: \begin{DoxyVerb}          - Controlador: sirve como intermediario entre una determinada interfaz y el algoritmo que la implementa, de tal forma que es el controlador quien recibe los datos del usuario y quien los envía a las distintas clases según el método llamado.
          - Chain of responsability: evita acoplar el emisor de una petición a su receptor dando a más de un objeto la posibilidad de responder a una petición.
\end{DoxyVerb}
 
\end{DoxyItemize}